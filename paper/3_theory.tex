% Escalating commitment staw


Within the logic of rational choice theory, \textit{escalation of commitment refers} to the irrational decision to allocate additional  resources to one decisional alternative over another 
Investment decisions =\enquote{situations in which resources are allocated to one decisional alternative over others} \parencite[p. 28]{Staw1976}


Prospect theory: \enquote{Loss-framed decisions (which include escalation decisions) leading to loss-aversion, and thus risk-seeking behavior (in this case further expenditures)} \parencite[][p. 544]{Sleesman2012}


% theoretical distinction from gambler's fallacy

% theoretical distinction from status-quo bias

Notes:

Difference between status quo bias, gambler's fallacy

Take into account that people need to feel a degree of personal responsibility (if post-hoc rationalization is the relevant mechanism , \enquote{self-justification  may similarly depend upon the level of personal responsibility one has had in determining a particular course of action and the outcomes resulting from those actions} ) \parencite[][p. 30]{Staw1976}