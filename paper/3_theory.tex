% Escalating commitment staw


Within the logic of rational choice theory, \textit{escalation of commitment refers} to the irrational decision to allocate additional  resources to one decisional alternative over another 
Investment decisions =\enquote{situations in which resources are allocated to one decisional alternative over others} \parencite[p. 28]{Staw1976}


Prospect theory: \enquote{Loss-framed decisions (which include escalation decisions) leading to loss-aversion, and thus risk-seeking behavior (in this case further expenditures)} \parencite[][p. 544]{Sleesman2012}


% theoretical distinction from gambler's fallacy

% theoretical distinction from status-quo bias

Notes:

Difference between status quo bias, gambler's fallacy

Take into account that people need to feel a degree of personal responsibility (if post-hoc rationalization is the relevant mechanism , \enquote{self-justification  may similarly depend upon the level of personal responsibility one has had in determining a particular course of action and the outcomes resulting from those actions} ) \parencite[][p. 30]{Staw1976}

%------------------------------------------------------
Subjective expected utility \parencite[][]{Savage1972}


%------------------------------------------------------
Self-justification Theory (cognitive dissonance) \parencite[][]{Aronson1968}\parencite[][]{Festinger1957}


%------------------------------------------------------
Prospect Theory \parencite[][]{Kahneman1979}


%------------------------------------------------------
Goal substitution \parencite[][]{Conlon1993}


%------------------------------------------------------
Self-presentation theory \parencite[][]{Goffman1959}\parencite[][]{Jones1982}



%------------------------------------------------------
Principal-agent theory \parencite[][]{Eisenhardt1989}\parencite[][]{Jensen1976} \parencite[][]{Ross1973}



%------------------------------------------------------
Project determinants
Mostly driven by subjective expected utility \parencite[][p. 542]{Sleesman2012 } \textcite[see][]{Staw1987}


Large infrastructure projects vs social programs touches several mechanisms

\begin{tblr}{
    width=\linewidth,
    colspec={XXXX},
    vline{2,3,4}={},
    hline{1-Z}={2-Z}{solid},
    }
    Category& Theory & Mechanism & Do they differ in this regard?\\
    Project determinants & Subjective expected utility & Decision risk & [X] \\
    Project determinants & Subjective expected utility & Opportunity cost information &  [] \\
    Project determinants & Subjective expected utility & Information set (information acquisition, decision uncertainty) &  [] \\
    Project determinants & Subjective expected utility & Positive performance trend information &  [] \\
    Project determinants & Subjective expected utility & Expressed preference for initial decision &  [] \\
    Psychological & Self-justification theory & Previous resource expenditure (sunk cost (money and time)) &  [] \\
    Psychological & Self-justification theory & Familiarity with decision context (expertise, self-confidence/efficacy) &  [] \\
    Psychological & Self-justification theory & Personal responsibility &  [] \\
    Psychological & Self-justification theory & Ego threat &  [X] \\
    Psychological & Prospect theory & Information framing &  [X] \\
    Psychological & Goal substitution & Proximity to project completion &  [X] \\
    Social Determinants & Self-presentation theory & Public evaluation of decision &  [] \\
    Social Determinants & Self-presentation theory & Resistance decision from others &  [] \\
    Social Determinants & Self-presentation theory & Group identity &  [X] \\
    Structural Determinants & Principal-agent theory & Agency problems &  [X] \\
\end{tblr}