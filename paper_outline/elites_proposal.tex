
\setcounter{page}{1}
\normalsize
\setstretch{1.3}



%Introduction------------------------------
\section*{Motivation and research question}

A number of large rail projects in western developed democracies, namely Stuttgart 21 (Germany), HS2 (UK) and California high-speed rail (US), have become somewhat notorious for cost overruns and delays. Construction continues in Germany and California, however recently HS2 has been substantially scaled back \parencite{May2023}. These projects are often cited as instances of \textit{sunk-cost} thinking in politics, i.e. continuing to fund suboptimal or failed policies, so that funds that have already been invested do not go to \textit{waste}. \textcite{Sheffer2017}, for instance, find in a series of survey experiments that politicians tend to exhibit several decision biases at similar rates as the general population. As regards the sunk-cost fallacy and escalation of commitment, however, politicians are actually more likely to exhibit biased thinking than the general population. 

\textcite[][]{Sheffer2017} ask respondents whether they would invest an additional \$100 million in a small business loan program, on which \$500 million have already been spent. Respondent are further informed that the program will not cover its cost, unless the additional money is spent (including the additional funds). However, for large infrastructure projects, the idea of financial solvency does not straightforwardly apply, e.g. a high-speed rail project is not expected to pay for itself directly, but diffusely via increased economic activity, emissions savings, etc. This raises the possibility that politicians may evaluate infrastructure programs run over budget differently from e.g. social security programs that run a deficit (like many pension systems do). I therefore ask:


\begin{quote}
  How does policy domain affect the incidence of sunk-cost fallacy thinking in politicians?
\end{quote}
 
\section*{Hypotheses}
One explanation for the sunk-cost fallacy is found in prospect theory \parencites{Kahnemann1979}{Arkes1985}, which states that individuals perceive themselves to be in the domain of losses, they take more risks. In this case, an issue's framing will determine occurrence of sunk-cost thinking. Cost overruns for infrastructure projects are not framed as a loss, rather, the true costs were wrongly estimated at the time the project was approved. At the same time, pension programs may be framed as fundamentally financially unsustainable. In this case one would expect:

\begin{quote}
  H1: Politicians will more exhibit less sunk-cost thinking when presented with budget overruns of large infrastructure projects compared to other kinds of spending.
\end{quote}


One explanation for the sunk-cost fallacy is found in prospect theory \parencites{Kahnemann1979}{Arkes1985}, which states that individuals perceive themselves to be in the domain of losses, they take more risks. In this case, an issue's framing will determine occurrence of sunk-cost thinking. Cost overruns for infrastructure projects are not framed as a loss, rather, the true costs were wrongly estimated at the time the project was approved. At the same time, pension programs may be framed as fundamentally financially unsustainable. In that case, one would expect people to exhibit less sunk-cost thinking when presented with budget overruns of large infrastructure projects compared to unsustainable social programs. O
