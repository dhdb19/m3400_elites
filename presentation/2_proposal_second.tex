%----------------------------------------------------
\begin{frame}{Research proposal}

\begin{columns}
  \begin{column}[T]{0.475\textwidth}
    \textbf{\color{kon4}Theory}

    \vspace*{0.2cm}

  \textcite[][]{Sleesman2012} deduce 4 categories of mechanisms (total 16):
  \begin{itemize}
    \item \textbf{Project}: Subjective expected utility -> Decision risk
    \item \textbf{Psychological}: Prospect theory -> Information framing (\textit{my hypothesis})
    \item \textbf{Social}: Self-presentation theory: No relevant dimensions
    \item \textbf{Structural}: Principal-agent (traditionally)
  \end{itemize}

% Subjective expected utility

% Self-justification Theory

% Prospect Theory

% Goal substitution

% Self-presentation theory 

% Principal-agent theory

    
  \end{column}

  \begin{column}[T]{0.475\textwidth}

    \textbf{\color{kon4}Research Design}

    \vspace*{0.2cm}

    \textbf{Survey vignette}: Annual budget vote is coming up. The [infrastructure project] / [social project] is up for renewal.

    \textbf{Infrastructure project manipulation}: Project is over-budget and will be late

    \vspace{0.2cm}

    \textbf{Social program manipulation}: Additional cash-infusion needed to keep the program solvent 

    \vspace{0.2cm}

    \textbf{Case selection}: National level lawmakers in Belgium,Canada, Germany, Israel, Switzerland,  


    \vspace{0.2cm}
  \end{column}
\end{columns}


\end{frame}
%-----------------------------------------------------

%------------------------------------------------------
\begin{frame}{Questions}
  
  Implement a survey vignette design that accounts for more/all possible mechanisms? 
  \begin{itemize}
    \item[+] More nuanced findings
    \item[-] Very complex vignettes -> failure to treat subjects in the intended way
  \end{itemize}

\end{frame}